We have organised our code in two main folders called \texttt{finDif} and \texttt{aux}. The first folder contains the main files of the Crank Nicolson method, \texttt{CrankNicolson.cpp} and \texttt{CrankNicolson.h} and two more files, \texttt{taks} and \texttt{aux}. The former contains the especific classes for solving both tasks, whist the latter contains auxiliar functions for the Crank Nicolson Method such that the set up of the Convertible Bond, \texttt{ConvertibleBonds.cpp} and\texttt{ConvertibleBonds.h} or functions for constructing the grid, \texttt{GridConstructos.cpp} and \texttt{GridConstructor.h}.

Within the \texttt{aux} function we can find files for printing macros and other general functions.

\subsubsection{\texttt{main.cpp}}
Here we provide the \texttt{main.cpp}. Most of functions are commented. To print something one just has to remove the \texttt{//}. 
\begin{lstlisting}
	#include <cmath>
	#include <vector>
	#include <algorithm>
	
	#include <iostream>
	#include <fstream>
	// For this task
	#include "CrankNicolson.h"
	#include "ConvertibleBonds.h"
	#include "Q1.h"
	#include "Q2.h"
	// General
	#include "GeneralFunctions.h"
	// To calculate the time
	#include <chrono>
	#define  CHRONO   std::chrono
	#define  SET_TIME CHRONO::system_clock::now()
	#define  DURATION CHRONO::duration
	#define  MILLI    std::milli
	
	#define NEXT_CASE cout << "#######" << endl;
	
	using namespace std;
	/* Template code for the Crank Nicolson Finite Difference
	*/
	int main(){
		cout.precision(15);
		// PARAMETERS
		double T = 3., F = 35., R = 2, r = 0.0117, kappa = 0.0833333333333, mu = 0.0053, X = 17.38, C = 0.205, alpha = 0.02, beta = 0.808, sigma = 0.66;
		double Smax = 250;
		double I = 500;
		double J = 500;
		ofstream output;
		Q1 q1(T, F, R, r, kappa, mu, X, C, alpha, beta, sigma, 1000, I, J);
		Q2 q2(T, F, R, r, kappa, mu, X, C, alpha, beta, sigma, 1000, I, J);
		/*
		TASK 1
		*/
		//    CONV_BONDS convBond(T, F, R, r, kappa, mu, X, C, alpha, beta, sigma, Smax, J, I);
		//    cout << (F - convBond.B(0)) / convBond.A(0) << endl;
		
		///
		/// FIXING BETA AND MOVING SIGMA
		///
		// vector <double> sigmas = {0, 0.381, 0.66, 0.802, 0.962, 1};
		// q1.variousSigmas(sigmas, THOMAS);
		
		///
		/// FIXING SIGMA AND MOVING BETA
		///
		// vector <double> betas = {0.892};
		// q1.variousBetas(betas, THOMAS);
		
		///
		/// FIXING SIGMA AND AND BETA AND MOVING KAPPA
		///
		// vector <double> kappas = {0, 0.02, 0.0833, 0.2, 0.5, 1};
		// q1.variousKappas(kappas, THOMAS);
		
		///
		/// INCREASING IMAX AND JMAX (SQUARE) AND FIXING SMAX
		///
		//  q1.variousV_fixedS0(17.38, 2, 2, 20000, 16, 8.);
		//  vector<double> v = {42.0495722630, 42.0495647830, 42.0495620065};
		//  q1.V_fixedS0(17.38, 16, 8., 952, 952);
		//  q1.V_fixedS0(17.38, 16, 8., 9216, 9216);
		//  AUX::extrap(v[1], v[2], 2);
		
		/*
		TASK 2
		*/
		//    q1.increasingS(500, 500, 300, 1.05);
		//    q2.increasingS(100, 100, 200, 1.05);
		//    q2.increasingS(200, 200, 200, 1.05);
		//    vector<double> rs = {0.00585, 0.0117, 0.01755};
		//    q2.variousInterestRates(rs, 300, 300, 200);
		//    q2.variousV_fixedS0(17.38, 0.96151, 2, 2, 80000, 16, 8.);
		//    q2.V_fixedS0(17.38, 0.96151, 16, 8., 128, 64);
		//    q2.V_fixedS0(17.38, 0.96151, 16, 8., 256, 128);
		//    AUX::extrap(43.18821969729999, 43.2219325264, 2);
		return 0;
	}
\end{lstlisting}

\subsubsection{\texttt{finDif} folder}

\begin{lstlisting}
	//
	//  CrankNicolson.h
	//  assignment2_codes
	//
	//  Created on 28/4/21.
	//
	
	#include <vector>
	#include <iostream>
	#include <fstream>
	using namespace std;
	
	#define CN  CCrankNicolson
	#define LU      1
	#define SOR     2
	#define THOMAS  3
	
	#define PSOR                1
	#define POLICY_ITERATION    2
	#define PENALTY             3
	
	#define EUROPEAN false
	#define AMERICAN true
	
	#define SAVE        true
	#define DONT_SAVE   false
	
	class CCrankNicolson
	{
		public:
		CCrankNicolson(double T, double F, double R, double r, double kappa, double mu, double X, double C, double alpha, double beta, double sigma, double S0, double Smax, long I, long J);
		// Main functions
		void eurConvertibleBond(ofstream *output, int method = 0, int degree = 2, double tol = 1.e-2, double omega = 1.2);
		void amConvertibleBond_penalty(ofstream *output, int degree=2, bool saveData = true, double tol = 1.e-6);
		// Get Option Value
		inline double GetV() {return m_optionValue;};
		private:
		// Useful Functions
		double approxPrice(vector<double> &v, vector<double> &s);
		vector<double> thomasSolve(const vector<double> &a,const vector<double> &b_,const vector<double> &c, vector<double> &d);
		// PARAMETERS
		double m_T;
		double m_F;
		double m_R;
		double m_r;
		double m_kappa;
		double m_mu;
		double m_X;
		double m_C;
		double m_alpha;
		double m_beta;
		double m_sigma;
		// AMERICAN PARAMETERS
		double m_P;
		double m_rho;
		double m_iterMax;
		double m_t0;
		// AUXILIAR PARAMETERS
		double m_kappar;
		double m_alphar;
		// Local Variables (dS, dt)
		double m_S0;
		double m_Smax;
		double m_dS;
		double m_dt;
		// Grid Parameters
		long   m_J;
		long   m_I;
		int    m_jStar;
		// Option Value
		double m_optionValue;
	};
\end{lstlistings}

\begin{lstlisting}
	//
	//  CrankNicolson.cpp
	//  assignment2_codes
	//
	//  Created  on 28/4/21.
	//
	
	/// Header of the explicit difference method
	#include "CrankNicolson.h"
	
	/// Header with convertible bonds functions
	#include "ConvertibleBonds.h"
	
	/// Header with some important and useful functions
	#include "GeneralFunctions.h"
	
	/// Header with payoff functions of several options
	#include "PayoffFunctions.h"
	
	/// Header for constructing the grid of the explicit difference method
	#include "GridConstructor.h"
	
	/// Header for printing data
	#include "PrintMacros.h"
	
	// To calculate the time
	#include <chrono>
	#define  CHRONO   std::chrono
	#define  SET_TIME CHRONO::system_clock::now()
	#define  DURATION CHRONO::duration
	#define  MILLI    std::milli
	
	/// Important libraries
	#include <cmath>
	#include <algorithm>
	#include <vector>
	#include <iostream>
	#include <fstream>
	using namespace std;
	
	/// First declare the parameters for the problem, local grid and variables and the vectors required
	/// In option pricing problems, we must for S over the semi infinite domain, therefore numerically we need to choose an appropiate Smax
	/// The first five parameters are intrinsice option parameters, whilst the last three ones are those appropriate parameters for the method.
	
	CN::CCrankNicolson(double T, double F, double R, double r, double kappa, double mu, double X, double C, double alpha, double beta, double sigma, double S0, double Smax, long I, long J){
		m_T         = T;
		m_F         = F;
		m_R         = R;
		m_r         = r;
		m_kappa     = kappa;
		m_mu        = mu;
		m_X         = X;
		m_C         = C;
		m_alpha     = alpha;
		m_beta      = beta;
		m_sigma     = sigma;
		m_kappar    = kappa + r;
		m_alphar    = alpha + r;
		m_S0        = S0;
		m_Smax      = Smax;
		m_J         = J;
		m_I         = I;
		m_dS       = m_Smax / J;
		m_dt        = m_T / I;
		m_jStar     = m_S0/m_dS;
	}
	void
	CN::eurConvertibleBond(ofstream *output, int method, int degree, double tol, double omega)
	{
		// CONVERTIBLE BONDS LIBRARY
		CONV_BONDS convBonds(m_T, m_F, m_R, m_r, m_kappa, m_mu, m_X, m_C, m_alpha, m_beta, m_sigma, m_Smax, m_I, m_J);
		//
		vector <double> vOld(m_J+1), vNew(m_J+1);
		// Setup and initialise the stock price
		vector <double> S = GRID::setupStockPrices(m_dS, m_J);
		// Setup and initialise the final conditions on the option price
		for (int j=0; j<=m_J; j++)
		{
			vOld[j] = PAYOFF::convBond(S[j], m_R, m_F);
			vNew[j] = PAYOFF::convBond(S[j], m_R, m_F);
			// PRINT_5DATA_LINE(m_I, j, S[j], vNew[j], vOld[j])
		}
		// start looping through time levels
		for(long i=m_I-1; i>=0; i--)
		{
			double t        = i * m_dt;
			double approx_t = (i + 0.5) * m_dt;
			/// BOUNDARY CONDITIONS
			// Declare vectors for matrix equations
			vector<double> a = {0.}, b = {1.}, c = {0.}, d = {convBonds.V_S0(approx_t)};
			// LU method
			vector<double> beta = {b[0]}, D = {d[0]};
			
			/// SET UP MATRIX EQUATIONS
			// PRINT_4DATA_LINE("a", "b", "c", "d")
			// PRINT_4DATA_LINE(a[0], b[0],c[0], d[0])
			for(int j=1;j< m_J;j++)
			{
				a.push_back(convBonds.aFunc(i, j));
				b.push_back(convBonds.bFunc(i, j));
				c.push_back(convBonds.cFunc(i, j));
				d.push_back(convBonds.dFunc(i, j, vOld));
				// LU method
				if (method == LU)
				{
					beta.push_back(convBonds.betaFunc(approx_t, j, beta[j-1]));
					D.push_back(convBonds.DFunc(approx_t, j, beta[j-1], d[j], D[j-1]));
				}
			}
			// Boundary conditions at S = Smax
			a.push_back(0.);
			b.push_back(1.);
			c.push_back(0.);
			d.push_back(convBonds.V_Smax(m_Smax,approx_t));
			
			// EUROPEAN OPTIONS
			// SOLVE WITH LU METHOD
			if (method == LU)
			{
				beta.push_back(convBonds.betaFunc(i, m_J, beta[m_J-1]));
				D.push_back(convBonds.DFunc(i, m_J, beta[m_J-1], d[m_J], D[m_J-1]));
				// Solve matrix equations with LU method
				vNew[0] = m_X;
				vNew[m_J] = D[m_J] / beta[m_J];
				for (long j=m_J-1; j >=0; j--)
				vNew[j] = convBonds.prevV(t, j, beta, D, vNew);
			}
			// SOLVE WITH SOR METHOD
			if (method == SOR)
			{
				// Solve matrix equations with SOR
				long y, sor, iterMax = 10000;
				for (sor = 0; sor < iterMax; sor++)
				{
					// SOR equations in here
					// j = 0
					{
						y = (d[0] - c[0] * vNew[1]) / b[0];
						vNew[0] = vNew[0] + omega * (y-vNew[0]);
					}
					// 0 < j < jMax
					for(long j=1; j<m_J; j++)
					{
						y = (d[j] - a[j] * vNew[j-1] - c[j]*vNew[j+1]) / b[j];
						vNew[j] = vNew[j] + omega * (y-vNew[j]);
					}
					// j = jMax
					{
						y = (d[m_J] - a[m_J] * vNew[m_J-1]) / b[m_J];
						vNew[m_J] = vNew[m_J] + omega * (y-vNew[m_J]);
					}
					// Calculate residual
					double error=0.;
					error += fabs(d[0] - b[0] * vNew[0] - c[0] * vNew[1]);
					for(long j=1; j<m_J;j++)
					error += fabs(d[j] - a[j]*vNew[j-1] - b[j]*vNew[j] - c[j]*vNew[j+1]);
					error += fabs(d[m_J] - a[m_J]*vNew[m_J-1] - b[m_J]*vNew[m_J]);
					// Check for convergence and exit loop if converged
					if(error < tol)
					break;
				}
				if(sor==iterMax)
				PRINT_DATA_LINE("NOT CONVERGED");
			}
			// SOLVE WITH THOMAS SOLVER ALGORITHM
			if (method == THOMAS)
			vNew = thomasSolve(a, b, c, d);
			// Set old=new
			vOld = vNew;
		}
		// Finish looping through time levels
		// output the estimated option price
		double optionValue = GRID::lagrangeInterpolation(vNew, S, m_S0, degree);//approxPrice(vNew, S);
		// output the estimated option price
		double asympV = m_S0*convBonds.A(0) + convBonds.B(0);
		DATA_LINE(output, m_F, m_I, m_J, m_S0, optionValue, asympV);
		PRINT_DATA_LINE(m_F, m_I, m_J, m_S0, optionValue, asympV);
		
	}
	
	void
	CN::amConvertibleBond_penalty(ofstream *output, int degree, bool saveData, double tol)
	{
		
		m_P = 38.;
		m_t0 = 0.96151;
		m_iterMax = 10000;
		m_rho = 1e8;
		
		// CONVERTIBLE BONDS LIBRARY
		CONV_BONDS convBonds(m_T, m_F, m_R, m_r, m_kappa, m_mu, m_X, m_C, m_alpha, m_beta, m_sigma, m_Smax, m_I, m_J);
		//
		vector <double> vOld(m_J+1), vNew(m_J+1);
		// Setup and initialise the stock price
		vector <double> S = GRID::setupStockPrices(m_dS, m_J);
		// Setup and initialise the final conditions on the option price
		for (long j=0; j<=m_J; j++)
		{
			vOld[j] = PAYOFF::convBond(S[j], m_R, m_F);
			vNew[j] = PAYOFF::convBond(S[j], m_R, m_F);
			// PRINT_5DATA_LINE(m_I, j, S[j], vNew[j], vOld[j])
		}
		// start looping through time levels
		for(long i=m_I-1; i>=0; i--)
		{
			double t        = i*m_dt;
			double approx_t = (i + 0.5) * m_dt;
			/// BOUNDARY CONDITIONS
			// Declare vectors for matrix equations
			vector<double> a = {0.}, b = {1.}, c = {0.}, d;
			if (approx_t <= m_t0)
			d = {max(m_P,convBonds.V_S0(approx_t))};
			else
			d = {convBonds.V_S0(approx_t)};
			/// SET UP MATRIX EQUATIONS
			for(int j=1;j< m_J;j++)
			{
				a.push_back(convBonds.aFunc(i, j));
				b.push_back(convBonds.bFunc(i, j));
				c.push_back(convBonds.cFunc(i, j));
				d.push_back(convBonds.dFunc(i, j, vOld));
			}
			// Boundary conditions at S = Smax
			a.push_back(0.);
			b.push_back(1.);
			c.push_back(0.);
			// d.push_back(convBonds.V_Smax(m_Smax,approx_t));
			d.push_back(m_R*m_Smax);
			// Temporal restriction
			int penaltyIt;
			for (int penaltyIt=0; penaltyIt < m_iterMax; penaltyIt++)
			{
				// Create new vectors containing a copy of the FD approx
				vector<double> aHat(a), bHat(b), cHat(c), dHat(d);
				// Apply penalty here to finite difference scheme
				for (int j=1; j<m_J; j++)
				{
					// If current value suggesta apply penalty, adjust matrix equations
					if(vNew[j] < m_R*S[j])
					{
						bHat[j] = b[j] + m_rho;
						dHat[j] = d[j] + m_rho*(m_R*S[j]);
					}
					if( approx_t < m_t0 && vNew[j] < m_P)
					{
						bHat[j] = bHat[j] + m_rho;
						dHat[j] = dHat[j] + m_rho*m_P;
					}
					
				}
				// Solve with thomas Method
				vector <double> y = thomasSolve(aHat, bHat, cHat, dHat);
				// y now contains next guess at solution
				// Check for differences between vNew and y
				double error = 0.;
				for (int j=0; j<= m_J; j++)
				error += (vNew[j] - y[j])*(vNew[j] - y[j]);
				// Update Value of vNew
				vNew = y;
				// make an exit condition when solution is converged
				if (error < tol * tol)
				break;
			}
			if(penaltyIt >= m_iterMax)
			{
				PRINT_DATA_LINE("Error NOT converging within required iterations");
				throw;
			}
			vOld=vNew;
		} // Ending temporal loop
		
		// Finish looping through time levels
		// output the estimated option price
		m_optionValue = GRID::lagrangeInterpolation(vNew, S, m_S0, degree);//approxPrice(vNew, S);
		// output the estimated option price
		double asympV = m_S0*convBonds.A(0) + convBonds.B(0);
		if (saveData == SAVE)
		{
			DATA_LINE(output, m_F, m_I, m_J, m_S0, m_optionValue, asympV);
		}
		PRINT_DATA_LINE(m_F, m_I, m_J, m_S0, m_optionValue, asympV);
	}
	
	
	// USEFUL FUNCTIONS
	double
	CN::approxPrice(vector<double> &v, vector<double> &s)
	{
		vector<double> p1 = {s[m_jStar], v[m_jStar]};
		vector<double> p2 = {s[m_jStar+1], v[m_jStar+1]};
		return AUX::lagInterp(m_S0, p1, p2);
	}
	
	vector<double>
	CN::thomasSolve(const vector<double> &a,const vector<double> &b_,const vector<double> &c, vector<double> &d)
	{
		auto n=a.size();
		std::vector<double> b(n),temp(n);
		// initial first value of b
		b[0]=b_[0];
		for(auto j=1;j<n;j++)
		{
			b[j]=b_[j]-c[j-1]*a[j]/b[j-1];
			d[j]=d[j]-d[j-1]*a[j]/b[j-1];
		}
		// calculate solution
		temp[n-1]=d[n-1]/b[n-1];
		for(int j=n-2; j>=0; j--)
		temp[j]=(d[j]-c[j]*temp[j+1])/b[j];
		return temp;
	}
	
	//if (method == PSOR){
		//    // Solve matrix equations with SOR
		//    int y, sor, iterMax = 10000;
		//    for (sor = 0; sor < iterMax; sor++)
		//    {
			//        // SOR equations in here
			//        // j = 0
			//        {
				//          y = (d[0] - c[0] * vNew[1]) / b[0];
				//          vNew[0] = max(vNew[0] + omega * (y-vNew[0]), m_P);
				//        }
			//        // 0 < j < jMax
			//        for(int j=1; j<m_J; j++)
			//        {
				//          y = (d[j] - a[j] * vNew[j-1] - c[j]*vNew[j+1]) / b[j];
				//          vNew[j] = max(vNew[j] + omega * (y-vNew[j]), m_P);
				//        }
			//        // j = jMax
			//        {
				//          y = (d[m_J] - a[m_J] * vNew[m_J-1]) / b[m_J];
				//          vNew[m_J] = max(vNew[m_J] + omega * (y-vNew[m_J]), m_P);
				//        }
			//        // Calculate residual
			//        double error=0.;
			//        error += fabs(d[0] - b[0] * vNew[0] - c[0] * vNew[1]);
			//        for(int j=1; j<m_J;j++)
			//          error += fabs(d[j] - a[j]*vNew[j-1] - b[j]*vNew[j] - c[j]*vNew[j+1]);
			//        error += fabs(d[m_J] - a[m_J]*vNew[m_J-1] - b[m_J]*vNew[m_J]);
			//        // Check for convergence and exit loop if converged
			//        if(error < tol)
			//          break;
			//    }
		//}
\end{lstlisting}

\begin{lstlisting}
	//
	//  PayoffFunctions.hpp
	//  ComputationalFinance
	//
	//  Created on 14/4/21.
	//
	
	#define  PAYOFF CPayoffFunctions
	
	class CPayoffFunctions{
		public:
		static double callOption(double stock_price, double strike_price);
		static double putOption(double stock_price, double strike_price);
		static double convBond(double stock_price, double R, double F);
	};
\end{lstlisting}
\begin{lstlisting}
	//
	//  PayoffFunctions.cpp
	//  ComputationalFinance
	//
	//  Created on 14/4/21.
	//
	
	#include "PayoffFunctions.h"
	#include "GeneralFunctions.h"
	
	// CALL OPTION PAYOFF
	double
	PAYOFF::callOption(double stock_price, double strike_price){
		return AUX::maxFunc(stock_price - strike_price, 0.);
	}
	
	// PUT OPTION PAYOFF
	double
	PAYOFF::putOption(double stock_price, double strike_price){
		return AUX::maxFunc(strike_price - stock_price, 0.);
	}
	
	// CONVERTIBLE BOND
	double
	PAYOFF::convBond(double stock_price, double R, double F)
	{
		return AUX::maxFunc(F, R * stock_price);
	}	
\end{lstlisting}

\begin{lstlisting}
	//
	//  ConvertibleBonds.hpp
	//  assignment2_codes
	//
	//  Created on 28/4/21.
	//
	
	#define CONV_BONDS CConvertibleBonds
	
	#include <vector>
	#include <iostream>
	#include <fstream>
	using namespace std;
	
	class CConvertibleBonds
	{
		public:
		CConvertibleBonds(double T, double F, double R, double r, double kappa, double mu, double X, double C, double alpha, double beta, double sigma, double Smax, long I, long J);
		
		double V_Smax(double S, double t);
		double V_S0(double t);
		// AUXILIAR FUNCTIONS
		double A(double t);
		double B(double t);
		// COEFFICIENTS FOR THE PDE
		double aFunc(long i, long j);
		double bFunc(long i, long j);
		double cFunc(long i, long j);
		double dFunc(long i, long j, vector<double> &v);
		double theta(double t);
		// LU coefficients
		double betaFunc(long i, long j, double prevBeta);
		double DFunc(long i, long j, double prevBeta, double d, double prevD);
		double prevV(long i, long j, vector<double>& beta, vector<double>& D, vector<double>& V);
		private:
		// PARAMETERS
		double m_T;
		double m_F;
		double m_R;
		double m_r;
		double m_kappa;
		double m_mu;
		double m_X;
		double m_C;
		double m_alpha;
		double m_beta;
		double m_sigma;
		// PDE PARAMETERS
		double m_Smax;
		double m_J;
		double m_I;
		double m_dS;
		double m_dt;
		// AUXILIAR PARAMETERS
		double m_kappar;
		double m_alphar;
		
	};
\end{lstlisting}

\begin{lstlisting}
	//
	//  ConvertibleBonds.cpp
	//  assignment2_codes
	//
	//  Created on 28/4/21.
	//
	
	#include "ConvertibleBonds.h"
	#include <cmath>
	#include <algorithm>
	#include <vector>
	#include <iostream>
	#include <fstream>
	using namespace std;
	
	CONV_BONDS::CConvertibleBonds(double T, double F, double R, double r, double kappa, double mu, double X, double C, double alpha, double beta, double sigma, double Smax, long I, long J)
	{
		m_T         = T;
		m_F         = F;
		m_R         = R;
		m_r         = r;
		m_kappa     = kappa;
		m_mu        = mu;
		m_X         = X;
		m_C         = C;
		m_alpha     = alpha;
		m_beta      = beta;
		m_sigma     = sigma;
		m_kappar    = kappa + r;
		m_alphar    = alpha + r;
		m_Smax      = Smax;
		m_J         = J;
		m_dS        = m_Smax / m_J;
		m_I         = I;
		m_dt        = m_T / m_I;
	}
	
	double
	CONV_BONDS::A(double t)
	{
		return m_R * exp(-m_kappar * (m_T - t));
	}
	
	double
	CONV_BONDS::B(double t)
	{
		double X_term = 2. * m_X * m_R * exp( - (m_r + 0.5 * m_kappa) * (m_T - t) ) * sinh( 0.5 * m_kappa * (m_T -t) );
		double C_term = - m_C / m_alphar * exp(m_r * t) * ( exp(-m_alphar * m_T) - exp(-m_alphar * t));
		return X_term + C_term;
	}
	
	// GENERAL VALUE FUNCTION
	double
	CONV_BONDS::V_Smax(double S, double t)
	{
		return S * A(t) + B(t);
	}
	
	// VALUE FUNCTION WHEN S = 0
	double
	CONV_BONDS::V_S0(double t)
	{
		double F_term = m_F * exp( -m_r * (m_T - t) );
		double C_term = m_C / ( m_r + m_alpha ) * exp( -m_alpha * m_T ) * ( exp(m_alpha * (m_T - t)) - exp(-m_r * (m_T - t)) );
		return F_term + C_term;
	}
	
	// PDE COEFFICIENTS
	double
	CONV_BONDS::aFunc(long i, long j)
	{
		double first_term, second_term;
		double approx_t     = (i + 0.5) * m_dt;
		first_term   = -0.25 * pow(m_sigma, 2.) * pow(j, 2.* m_beta) * pow(m_dS, 2*(m_beta-1));
		second_term  =  0.25 * m_kappa * (theta(approx_t) / m_dS - j);
		return first_term + second_term;
	}
	
	double
	CONV_BONDS::bFunc(long i, long j)
	{
		double long_term = 0.5 * pow(m_sigma, 2.) * pow(j, 2. * m_beta) * pow(m_dS, 2. * (m_beta - 1.));
		return 1. / m_dt + 0.5 * m_r  + long_term;
	}
	
	double
	CONV_BONDS::cFunc(long i, long j)
	{
		double first_term, second_term;
		double approx_t     = (i + 0.5) * m_dt;
		first_term   = -0.25 * pow(m_sigma, 2.) * pow(j, 2.*m_beta) * pow(m_dS, 2. * (m_beta-1.));
		second_term  = -0.25 * m_kappa * (theta(approx_t) / m_dS - j);
		return first_term + second_term;
	}
	
	double
	CONV_BONDS::dFunc(long i, long j, vector<double> &v)
	{
		double approx_t = (i + 0.5) * m_dt;
		double a = aFunc(i, j);
		double b = bFunc(i, j);
		double c = cFunc(i, j);
		double d = - (a * v[j-1] + (b - 2. / m_dt) * v[j] + c * v[j+1]) + m_C * exp(-m_alpha * approx_t);
		return d;
	}
	
	double
	CONV_BONDS::theta(double t)
	{
		return (1. + m_mu) * m_X * exp(m_mu * t);
	}
	
	// LU COEFFICIENTS
	double
	CONV_BONDS::betaFunc(long i, long j, double prevBeta)
	{
		return bFunc(i, j) - (aFunc(i, j) * cFunc(i, j-1)) / prevBeta;
	}
	double
	CONV_BONDS::DFunc(long i, long j, double prevBeta, double d, double prevD)
	{
		return d - (aFunc(i, j) * prevD) / prevBeta;
	}
	double
	CONV_BONDS::prevV(long i, long j, vector<double> &beta, vector<double> &D, vector<double> &V)
	{
		return 1. / beta[j] * (D[j] - cFunc(i, j) * V[j+1]);
	}
\end{lstlisting}

\begin{lstlisting}
	//
	//  GridConstructor.h
	//  ComputationalFinance
	//
	//  Created on 15/4/21.
	//
	
	#include <vector>
	#define  GRID CGridConstructor
	
	using namespace std;
	
	class CGridConstructor{
		public:
		static vector<double>   setupStockPrices(double dS, long J);
		static double           boundaryConditions(int j);
		static double           lagrangeInterpolation(const vector<double>& y,const vector<double>& x, double x0, unsigned int n);
	};
\end{lstlisting}
\begin{lstlisting}
	//
	//  GridConstructor.cpp
	//  ComputationalFinance
	//
	//  Created  on 15/4/21.
	//
	#include "GridConstructor.h"
	#include <vector>
	using namespace std;
	
	vector<double>
	GRID::setupStockPrices(double dS, long J){
		vector<double> S(J+1);
		for (int j=0; j<=J; j++){
			S[j]  = j * dS;
		}
		return S;
	}
	
	// A generic llgrange interpolation function
	double
	GRID::lagrangeInterpolation(const vector<double>& y,const vector<double>& x,double x0, unsigned int n)
	{
		if(x.size()<n)return lagrangeInterpolation(y,x,x0,x.size());
		if(n==0)throw;
		int nHalf = n/2;
		int jStar;
		double dx=x[1]-x[0];
		if(n%2==0)
		jStar = int((x0 - x[0])/dx) -(nHalf-1);
		else
		jStar = int((x0 - x[0])/dx+0.5)-(nHalf);
		jStar= max(0,jStar);
		jStar= min(int(x.size()-n),jStar);
		if(n==1)
		return y[jStar];
		double temp = 0.;
		for(unsigned int i=jStar;i<jStar+n;i++){
			double  int_temp;
			int_temp = y[i];
			for(unsigned int j=jStar;j<jStar+n;j++){
				if(j==i){continue;}
				int_temp *= ( x0 - x[j] )/( x[i] - x[j] );
			}
			temp += int_temp;
		}
		// end of interpolate
		return temp;
	}
\end{lstlisting}

\begin{lstlisting}
	//
	//  Q1.hpp
	//  assignment2_codes
	//
	//  Created on 6/5/21.
	//
	
	#define Q1 CQ1
	#include <vector>
	using namespace std;
	
	class CQ1
	{
		public:
		CQ1(double T, double F, double R, double r, double kappa, double mu, double X, double C, double alpha, double beta, double sigma, double Smax, int J, int I);
		// Main functions
		void variousBetas(vector<double> &betas, int method);
		void variousKappas(vector<double> &kappas, int method);
		void variousSigmas(vector<double> &sigmas, int method);
		void increasingS(int I, int J, int iterMax, double gap = 1.1,  double r = 0);
		void V_fixedS0(double S0, int deg, int timesX, int I, int J);
		void variousV_fixedS0(double S0, double increment, int nMin, int nMax, int deg, int timesX);
		private:
		// Useful Functions
		// PARAMETERS
		double m_T;
		double m_F;
		double m_R;
		double m_r;
		double m_kappa;
		double m_mu;
		double m_X;
		double m_C;
		double m_alpha;
		double m_beta;
		double m_sigma;
		// AUXILIAR PARAMETERS
		double m_kappar;
		double m_alphar;
		// Local Variables (dS, dt)
		double m_Smax;
		double m_dS;
		double m_dt;
		// Grid Parameters
		int    m_J;
		int    m_I;
		int    m_jStar;
	};
	
	// To calculate the time
	#include <chrono>
	#define  CHRONO   std::chrono
	#define  SET_TIME CHRONO::system_clock::now()
	#define  START_TIME CHRONO::system_clock::now()
	#define  END_TIME CHRONO::system_clock::now()
	#define  DURATION CHRONO::duration
	#define  MILLI    std::milli
\end{lstlisting}

\begin{lstlisting}
	//
	//  Q1.cpp
	//  assignment2_codes
	//
	//  Created on 6/5/21.
	//
	
	#include "Q1.h"
	#include <vector>
	#include <cmath>
	#include <algorithm>
	
	#include "CrankNicolson.h"
	#include "GeneralFunctions.h"
	
	// To keep results
	#define COMMA << "," <<
	#define END_LINE << endl;
	
	#include <iostream>
	#include <fstream>
	using namespace std;
	
	Q1::CQ1(double T, double F, double R, double r, double kappa, double mu, double X, double C, double alpha, double beta, double sigma, double Smax, int J, int I){
		m_T         = T;
		m_F         = F;
		m_R         = R;
		m_r         = r;
		m_kappa     = kappa;
		m_mu        = mu;
		m_X         = X;
		m_C         = C;
		m_alpha     = alpha;
		m_beta      = beta;
		m_sigma     = sigma;
		m_kappar    = kappa + r;
		m_alphar    = alpha + r;
		m_Smax      = Smax;
		m_J         = J;
		m_I         = I;
		m_dS        = m_Smax / J;
		m_dt        = m_T / I;
	}
	
	void
	Q1::variousBetas(vector<double> &betas, int method)
	{
		ofstream output;
		string strSigma = AUX::doubleToString(m_sigma);
		for (auto beta : betas)
		{
			string strBeta = AUX::doubleToString(beta);
			output.open("/Users/rubenexojo/Library/Mobile Documents/com~apple~CloudDocs/MSc Mathematical Finance - Manchester/subjects/II_semester/MATH60082_computational_finance/c++/assignment2/data/task1/eurConvBondValues_beta" + strBeta + "_sigma" + strSigma + ".csv");
			output << "F,I,J,S,V,VtoINF" << endl;
			for (double S = 2; S < m_Smax; S*=1.1)
			{
				CN crank(m_T, m_F, m_R, m_r, m_kappa, m_mu, m_X, m_C, m_alpha, beta, m_sigma, S, m_Smax, m_J, m_I);
				crank.eurConvertibleBond(&output, method);
			}
			output.close();
		}
	}
	void
	Q1::variousKappas(vector<double> &kappas, int method)
	{
		ofstream output;
		string strBeta  = AUX::doubleToString(m_beta);
		string strSigma = AUX::doubleToString(m_sigma);
		for (auto kappa : kappas)
		{
			string strKappa = AUX::doubleToString(kappa);
			output.open("/Users/rubenexojo/Library/Mobile Documents/com~apple~CloudDocs/MSc Mathematical Finance - Manchester/subjects/II_semester/MATH60082_computational_finance/c++/assignment2/data/task1/eurConvBondValues_beta" + strBeta + "_kappa" + strKappa + "_sigma" + strSigma + ".csv");
			output << "F,I,J,S,V,VtoINF" << endl;
			for (double S = 2; S < m_Smax; S*=1.1)
			{
				CN crank(m_T, m_F, m_R, m_r, kappa, m_mu, m_X, m_C, m_alpha, m_beta, m_sigma, S, m_Smax, m_J, m_I);
				crank.eurConvertibleBond(&output, method);
			}
			output.close();
		}
	}
	
	void
	Q1::variousSigmas(vector<double> &sigmas, int method)
	{
		ofstream output;
		string strBeta = AUX::doubleToString(m_beta);
		for (auto sigma : sigmas)
		{
			string strSigma = AUX::doubleToString(sigma);
			output.open("/Users/rubenexojo/Library/Mobile Documents/com~apple~CloudDocs/MSc Mathematical Finance - Manchester/subjects/II_semester/MATH60082_computational_finance/c++/assignment2/data/task1/eurConvBondValues_beta" + strBeta + "_sigma" + strSigma + ".csv");
			output << "F,I,J,S,V,VtoINF" << endl;
			for (double S = 2; S < m_Smax; S*=1.1)
			{
				CN crank(m_T, m_F, m_R, m_r, m_kappa, m_mu, m_X, m_C, m_alpha, m_beta, sigma, S, m_Smax, m_J, m_I);
				crank.eurConvertibleBond(&output, method);
			}
			output.close();
		}
	}
	
	void
	Q1::increasingS(int I, int J, int iterMax, double gap,  double r)
	{
		if (r == 0)
		{
			ofstream output;
			output.open("/Users/rubenexojo/Library/Mobile Documents/com~apple~CloudDocs/MSc Mathematical Finance - Manchester/subjects/II_semester/MATH60082_computational_finance/c++/assignment2/data/task1/eurConvBondValues_I" + to_string(I) + "_J" + to_string(J) + "_iterMax" + to_string(iterMax) + "_rho1e8.csv");
			output << "F,I,J,S,V,VtoINF" << endl;
			for (double S = 2; S < iterMax; S*=gap)
			{
				CN crank(m_T, m_F, m_R, m_r, m_kappa, m_mu, m_X, m_C, m_alpha, m_beta, m_sigma, S, m_Smax, J, I);
				crank.eurConvertibleBond(&output, THOMAS, 2);
			}
			output.close();
		}
		else
		{
			string strR = AUX::doubleToString(r);
			ofstream output;
			output.open("/Users/rubenexojo/Library/Mobile Documents/com~apple~CloudDocs/MSc Mathematical Finance - Manchester/subjects/II_semester/MATH60082_computational_finance/c++/assignment2/data/task1/eurConvBondValues_I" + to_string(I) + "_J" + to_string(J) + "_iterMax" + to_string(iterMax) + "_rho1e8r" + strR + ".csv");
			output << "F,I,J,S,V,VtoINF" << endl;
			for (double S = 2; S <= iterMax; S*=gap)
			{
				CN crank(m_T, m_F, m_R, r, m_kappa, m_mu, m_X, m_C, m_alpha, m_beta, m_sigma, S, m_Smax, J, I);
				crank.eurConvertibleBond(&output, THOMAS, 2);
			}
			output.close();
		}
	}
	
	void
	Q1::variousV_fixedS0(double S0, double incr, int nMin, int nMax, int deg, int timesX)
	{
		double Smax = timesX * m_X;
		ofstream output;
		string strDeg = to_string(deg);
		string strSmax= to_string(timesX);
		output.open("/Users/rubenexojo/Library/Mobile Documents/com~apple~CloudDocs/MSc Mathematical Finance - Manchester/subjects/II_semester/MATH60082_computational_finance/c++/assignment2/data/task1/eurConvBondValues_increasing_iMax_and_jMax_deg" + strDeg + "_Smax" + strSmax + "X.csv");
		output << "F,I,J,S,V,VtoInf" << endl;
		for (int n=nMin; n<=nMax; n*=incr)
		{
			CN crank(m_T, m_F, m_R, m_r, m_kappa, m_mu, m_X, m_C, m_alpha, m_beta, m_sigma, S0, Smax, n*ceil(Smax/S0), n*ceil(Smax/S0));
			crank.eurConvertibleBond(&output, THOMAS, deg);
		}
	}
	
	void
	Q1::V_fixedS0(double S0, int deg, int timesX, int I, int J)
	{
		// Convert the integers into strings to name the document
		ofstream output;
		// Open the document in order to keep results
		string strDeg = to_string(deg);
		string strSmax= to_string(timesX);
		output.open("/Users/rubenexojo/Library/Mobile Documents/com~apple~CloudDocs/MSc Mathematical Finance - Manchester/subjects/II_semester/MATH60082_computational_finance/c++/assignment2/data/task1/eurConvBondValue_" + strSmax + "X_deg" + strDeg + "_I" + to_string(I) + "_J" + to_string(J) + "_timing.csv");
		// First Line of the .csv file to further get columns of data
		output << "I,J,V,time" END_LINE
		auto start = START_TIME;
		// Produce results
		CN crank(m_T, m_F, m_R, m_r, m_kappa, m_mu, m_X, m_C, m_alpha, m_beta, m_sigma, S0, timesX * m_X, I, J);
		crank.eurConvertibleBond(&output, THOMAS);
		double optionValue = crank.GetV();
		// Keep the time of the end of the Crank Nicolson method
		auto end = END_TIME;
		// Duration: end - start
		DURATION<float> duration = (end - start);
		output << I COMMA J COMMA optionValue COMMA duration.count() END_LINE
		cout << "Our result has been obtained in " << duration.count() << "s" << endl;
	}
\end{lstlisting}

\begin{lstlisting}
	//
	//  Q2.hpp
	//  assignment2_codes
	//
	//  Created on 7/5/21.
	//
	
	#define Q2 CQ2
	#include <vector>
	using namespace std;
	
	class CQ2
	{
		public:
		CQ2(double T, double F, double R, double r, double kappa, double mu, double X, double C, double alpha, double beta, double sigma, double Smax, int J, int I);
		// Main functions
		void increasingS(int I, int J, int iterMax, double gap = 1.1, double r = 0);
		void variousInterestRates(vector<double> &rs, int I, int J, int iterMax, double gap = 1.1);
		void V_fixedS0(double S0, double t0, int deg, int timesX, int I, int J);
		void variousV_fixedS0(double S0, double t0, double increment, int nMin, int nMax, int deg, int timesX);
		private:
		//
		string doubleToString(double value, bool integer = false);
		// Useful Functions
		// PARAMETERS
		double m_T;
		double m_F;
		double m_R;
		double m_r;
		double m_kappa;
		double m_mu;
		double m_X;
		double m_C;
		double m_alpha;
		double m_beta;
		double m_sigma;
		
		// AUXILIAR PARAMETERS
		double m_kappar;
		double m_alphar;
		// Local Variables (dS, dt)
		double m_Smax;
		double m_dS;
		double m_dt;
		// Grid Parameters
		int    m_J;
		int    m_I;
		int    m_jStar;
	};
	
	// To calculate the time
	#include <chrono>
	#define  CHRONO   std::chrono
	#define  SET_TIME CHRONO::system_clock::now()
	#define  START_TIME CHRONO::system_clock::now()
	#define  END_TIME CHRONO::system_clock::now()
	#define  DURATION CHRONO::duration
	#define  MILLI    std::milli
\end{lstlisting}

\begin{lstlisting}
	//
	//  Q2.cpp
	//  assignment2_codes
	//
	//  Created on 7/5/21.
	//
	
	#include "Q2.h"
	#include <vector>
	#include <cmath>
	#include <algorithm>
	
	#include "CrankNicolson.h"
	#include "GeneralFunctions.h"
	
	// To keep results
	#define COMMA << "," <<
	#define END_LINE << endl;
	
	#include <iostream>
	#include <fstream>
	using namespace std;
	
	Q2::CQ2(double T, double F, double R, double r, double kappa, double mu, double X, double C, double alpha, double beta, double sigma, double Smax, int J, int I){
		m_T         = T;
		m_F         = F;
		m_R         = R;
		m_r         = r;
		m_kappa     = kappa;
		m_mu        = mu;
		m_X         = X;
		m_C         = C;
		m_alpha     = alpha;
		m_beta      = beta;
		m_sigma     = sigma;
		m_kappar    = kappa + r;
		m_alphar    = alpha + r;
		m_Smax  = Smax;
		m_J     = J;
		m_I     = I;
		m_dS    = m_Smax / J;
		m_dt    = m_T / I;
	}
	
	void
	Q2::increasingS(int I, int J, int iterMax, double gap, double r)
	{
		if (r == 0)
		{
			ofstream output;
			output.open("/Users/rubenexojo/Library/Mobile Documents/com~apple~CloudDocs/MSc Mathematical Finance - Manchester/subjects/II_semester/MATH60082_computational_finance/c++/assignment2/data/task2/amConvBondValues_I" + to_string(I) + "_J" + to_string(J) + "_iterMax" + to_string(iterMax) + "_rho1e8.csv");
			output << "F,I,J,S,V,VtoINF" << endl;
			for (double S = 2; S < iterMax; S*=gap)
			{
				CN crank(m_T, m_F, m_R, m_r, m_kappa, m_mu, m_X, m_C, m_alpha, m_beta, m_sigma, S, m_Smax, I, J);
				crank.amConvertibleBond_penalty(&output,2);
			}
			output.close();
		}
		else
		{
			string strR = AUX::doubleToString(r);
			ofstream output;
			output.open("/Users/rubenexojo/Library/Mobile Documents/com~apple~CloudDocs/MSc Mathematical Finance - Manchester/subjects/II_semester/MATH60082_computational_finance/c++/assignment2/data/task2/amConvBondValues_I" + to_string(I) + "_J" + to_string(J) + "_iterMax" + to_string(iterMax) + "_rho1e8r" + strR + ".csv");
			output << "F,I,J,S,V,VtoINF" << endl;
			for (double S = 2; S < iterMax; S*=gap)
			{
				CN crank(m_T, m_F, m_R, r, m_kappa, m_mu, m_X, m_C, m_alpha, m_beta, m_sigma, S, m_Smax, I, J);
				crank.amConvertibleBond_penalty(&output, 2);
			}
			output.close();
		}
	}
	
	void
	Q2::variousInterestRates(vector<double> &rs, int I, int J, int iterMax, double gap)
	{
		for (auto r : rs)increasingS(I, J, iterMax, gap, r);
	}
	
	void
	Q2::variousV_fixedS0(double S0, double t0, double increment, int nMin, int nMax, int deg, int timesX)
	{
		double Smax = timesX * m_X;
		ofstream output;
		// Convert the integers into strings to name the document
		string strDeg = to_string(deg);
		string strSmax= to_string(timesX);
		// Open the document in order to keep results
		output.open("/Users/rubenexojo/Library/Mobile Documents/com~apple~CloudDocs/MSc Mathematical Finance - Manchester/subjects/II_semester/MATH60082_computational_finance/c++/assignment2/data/task2/amConvBondValues_increasing_iMax_and_jMax_deg" + strDeg + "_Smax" + strSmax + "X.csv");
		// First Line of the .csv file to further get columns of data
		output << "F,I,J,S,V,VtoInf" << endl;
		// Iteration on the size of the grid. I and J are conveniently chosen in order to have S0 and t0 on the grid
		// or close to the grid.
		for (int n=nMin; n<=nMax; n*=increment)
		{
			// Produce and keep results
			CN crank(m_T, m_F, m_R, m_r, m_kappa, m_mu, m_X, m_C, m_alpha, m_beta, m_sigma, S0, Smax, n * ceil(Smax / S0), n * ceil(m_T / t0));
			crank.amConvertibleBond_penalty(&output, deg);
		}
	}
	
	void
	Q2::V_fixedS0(double S0, double t0, int deg, int timesX, int I, int J)
	{
		// Convert the integers into strings to name the document
		ofstream output;
		// Open the document in order to keep results
		string strDeg = to_string(deg);
		string strSmax= to_string(timesX);
		output.open("/Users/rubenexojo/Library/Mobile Documents/com~apple~CloudDocs/MSc Mathematical Finance - Manchester/subjects/II_semester/MATH60082_computational_finance/c++/assignment2/data/task2/amConvBondValue_" + strSmax + "X_deg" + strDeg + "_I" + to_string(I) + "_J" + to_string(J) + "_timing.csv");
		// First Line of the .csv file to further get columns of data
		output << "I,J,V,time" << endl;
		// Keep the time of inisiation of the Crank Nicolson method
		auto start = START_TIME;
		// Produce results
		CN crank(m_T, m_F, m_R, m_r, m_kappa, m_mu, m_X, m_C, m_alpha, m_beta, m_sigma, S0, timesX * m_X, I, J);
		crank.amConvertibleBond_penalty(&output, deg, DONT_SAVE);
		double optionValue = crank.GetV();
		// Keep the time of the end of the Crank Nicolson method
		auto end = END_TIME;
		// Duration: end - start
		DURATION<float> duration = (end - start);
		output << I COMMA J COMMA optionValue COMMA duration.count() END_LINE
		cout << "Our result has been obtained in " << duration.count() << "s" << endl;
	}
\end{lstlisting}
\subsubsection{\texttt{aux}}

\begin{lstlisting}
	//
	//  PrintMacros.h
	//  ComputationalFinance
	//
	//  Created  on 24/4/21.
	//
	
	#include <iostream>
	#include <fstream>
	#include <limits>
	using namespace std;
	
	#define COMMA               << ", ";
	#define START_LINE          cout <<
	#define END_LINE            << endl;
	#define OUTPUT              output <<
	
	#define END_PRINTING(a)     START_LINE (a) END_LINE
	#define PRINT(a)            START_LINE (a) COMMA
	
	#define END_WRITING(a)      OUTPUT (a) END_LINE
	#define WRITE(a)            OUTPUT fixed << setprecision(10) << (a) COMMA
	
	/// CONSOLE
	// RECURSION FOR PRINTING RESULTS IN THE CONSOLE: WE FIRSTLY DECLARE THE "LAST STATE" IN THE
	// RECURSION, I.E., WHEN THERE'S ONLY ONE ARGUMENT LEFT TO PRINTING.
	// THE SECOND TEMPLATE DOES THE REST.
	
	template<typename T>                    // Type is resolved in compile time
	void PRINT_DATA_LINE(T t)
	{
		END_PRINTING(t)
	}
	
	template<typename T, typename... ARGS>
	void PRINT_DATA_LINE(T t, ARGS... args) // Take the first arguments
	{
		PRINT(t)
		PRINT_DATA_LINE(args...);           // Recursion with a fewer argument
	}
	
	/// CSV
	// (SAME PROCEDURE THAN BEFORE)
	template<typename T>
	void DATA_LINE(ofstream *output, T t)
	{
		*END_WRITING(t)
	}
	
	template<typename T, typename... ARGS>
	void DATA_LINE(ofstream *output, T t, ARGS... args)
	{
		*WRITE(t)
		DATA_LINE(output, args...);
	}
	
	void OpenCSVFile(ofstream *output, string name, bool over_write){
		// open up a file stream to write data
		if (over_write == true)
		(*output).open("/Users/rubenexojo/Library/Mobile Documents/com~apple~CloudDocs/MSc Mathematical Finance - Manchester/subjects/II_semester/MATH60082_computational_finance/c++/assignment2/data/" + name + ".csv");
		if (over_write == false)
		(*output).open("/Users/rubenexojo/Library/Mobile Documents/com~apple~CloudDocs/MSc Mathematical Finance - Manchester/subjects/II_semester/MATH60082_computational_finance/c++/assignment2/data/" + name + ".csv", fstream::app);
		// check if the file is opened
		if (!(*output).is_open()){
			PRINT_DATA_LINE(" File not opened");
			// stop here
			throw;
		}
	}
\end{lstlisting}

\begin{lstlisting}
	//
	//  GeneralFunctions.h
	//  ComputationalFinance
	//
	//  Created on 14/4/21.
	//
	
	#include <vector>
	
	#define  AUX CGeneralFunctions
	
	using namespace std;
	
	class CGeneralFunctions{
		public:
		static double discountFactor(double interest_rate, double time);
		static double maxFunc(double x, double y);
		static double minFunc(double x, double y);
		static double lagInterp(double S, vector<double> p1, vector<double> p2);
		static string doubleToString(double value);
		static void   extrap(double S1, double S2, double p);
	};
	
	// To calculate the time
	#include <chrono>
	#define  CHRONO   std::chrono
	#define  SET_TIME CHRONO::system_clock::now()
	#define  START_TIME CHRONO::system_clock::now()
	#define  END_TIME CHRONO::system_clock::now()
	#define  DURATION CHRONO::duration
	#define  MILLI    std::milli
\end{lstlisting}

\begin{lstlisting}
	//
	//  GeneralFunctions.cpp
	//  ComputationalFinance
	//
	//  Created on 14/4/21.
	//
	
	#include "GeneralFunctions.h"
	#define  AUX CGeneralFunctions
	#define COMMA << "," <<
	#define END_LINE << endl;
	
	
	#include <cmath>
	#include <vector>
	#include <iostream>
	#include <fstream>
	using namespace std;
	
	double
	AUX::discountFactor(double interest_rate, double time)
	{
		return exp(- interest_rate * time);
	}
	double
	AUX::maxFunc(double x, double y)
	{
		if (x <= y)
		return y;
		else
		return x;
	}
	double
	AUX::minFunc(double x, double y)
	{
		if (x <= y)
		return x;
		else
		return y;
	}
	double
	AUX::lagInterp(double S, vector<double> p1, vector<double> p2)
	{
		double approxJ        = (S - p2[0]) / (p1[0] - p2[0]) * p1[1];
		double approxJplus1   = (S - p1[0]) / (p2[0] - p1[0]) * p2[1];
		return approxJ + approxJplus1;
	}
	string
	AUX::doubleToString(double value)
	{
		string str = to_string(value);
		if (value < 10)
		// erase the comma of a number lower than 10
		str = str.erase(1,1);
		if (value > 10 && value < 100)
		// erase the comma of a number greater than 10
		str = str.erase(2,2);
		// 4-5 decimals
		str = str.erase(5);
		return str;
	}
	
	void
	AUX::extrap(double S1, double S2, double p)
	{
		auto start = START_TIME;
		double extrapValue = (pow(2,p)*S2 - S1) / (pow(2,p) - 1);
		auto end   = END_TIME;
		DURATION<float> duration = (end - start);
		cout << extrapValue COMMA duration.count() END_LINE;
	}
\end{lstlisting}