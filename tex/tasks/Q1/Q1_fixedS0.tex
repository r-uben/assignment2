In this task we have to find the value of the option when $S_0 = 17.38$, and $\beta = 0.808$ and $\sigma  = 0.66$. To do so, we set a squared-grid, i.e., $\texttt{m_J} = \texttt{m_I} = N$ and obtain different values (first column of the next tables). From a first sight, we can roughly see that the value of the option when $S_0 = 17,38$ is around $42.05\pm10^{-2}$. However we would like to see how much we could improve our result to get more digits of precision. This can be done by increasing $S_{max}$. The idea is to have $S_0$ within the points of the grid, although we can make use of Lagrange Interpolation (of different degrees) to make sure we have a correct result. However, this would add some errors, and it is desiderable to avoid them. The idea is therefore to make sure that there exists a $j^*$ such that $S_0 = j^* \Delta S$, where $\Delta S = S_{max} / J$. Therefore, $j^* = \frac{S_0 J}{S_{max}}$ but, in our case, we have that $X = S_0$, so an easy way to make $J$ an integer is to take $S_{max} = k X$. This would imply that $J =  k j^*$ for some integer $k$. On the other hand, in order to study the convergence, we must iterate the method with different $J$. To do so, we have done a loop ``\texttt{for (int n=nMin; n<=nMax; n*=incr)}'' with $\texttt{nMin=3, nMax=50000, incr=2}$ (they could have been different) and with $\texttt{I=J=n*ceil(Smax/S0)=n*k}$. We can see an example in Table \ref{table1}.

\begin{table}[h!]\scriptsize
	\setlength{\tabcolsep}{5pt}
	\renewcommand{\arraystretch}{1.3}
	\begin{tabular}{c|ccccccccc}
		\hline\addlinespace[0.1cm]
		I=J			& 48	& 96 & 192 & 384 & 768 & 768 & 1546 & 2072 & 6144 \\
		$V(S_0,0)$ & 41.8408512 & 42.004891854 & 42.04124385 & 42.04867403 & 42.04973806 & 42.049609 & 42.04957 & 42.0495647 & 42.049562 
       \\\addlinespace[0.1cm]
		\hline
	\end{tabular}
	\vspace{0.3cm}
	\captionsetup{width=.55\linewidth}
	\caption{Values when increasing $\texttt{m_I}$ and $\texttt{m_J}$, Lagrange Interpolation of degree $16$ and $S_{max} = 8X$.}\label{table1}
\end{table}
\vspace{-0.6cm}
In order to improve our results and get more efficiency using a lower $J$, we can study the convergence of our values to further use Richardson Extrapolation (maybe more than one time). The ratios of the difference between results\footnote{Given three values $v_1$, $v_2$, $v_3$, we define the ratio between their differences as $\frac{v_2-v_1}{v_3-v_2}$. Since the convergence of Crank Nicolson method is $\mathcal{O}((\Delta S)^2, (\Delta t)^2)$, then we expect this ratio to be $4$, because the increments are produced by doubling $J$.} can be found in Table \ref{table2}.

\begin{table}[h]\scriptsize
	\setlength{\tabcolsep}{8pt}
	\renewcommand{\arraystretch}{1.3}
	\begin{tabular}{r|rrrrrrrrrr}
		\hline\addlinespace[0.1cm]
		$V(S_0,0)$ & 4.51256 & 4.89249 & 6.98301 &  8.26535 & 3.47294 & 4.95559 & 2.69404 & 9.72845 & 0.97506 & \\
		I Extrap & 2.80979 & 2.08924 & 2.01019 & 80.82043 & 2.73326 & 1.97126 & 2.21787 & 1.84651 & 2.29676 &          \\\addlinespace[0.1cm]
		\hline
	\end{tabular}
	\vspace{0.2cm}
	\captionsetup{width=.55\linewidth}
	\caption{Ratios from Table \ref{table1} and after doing one extrapolation with $p=2$.}\label{table2}
\end{table}
\vspace{-1cm}
As we can see, the ratios are not stricly monotonic, but the majority of them are near $4\pm1$, so we can try to do Richardson Extrapolation in order to try to get a more accurate result in less time. Richardson extrapolation provides, from a method of order $p$, an extrapolated result $v_{extrap}$ which can be found in the second row of Table~\ref{table2} and which is given by
\begin{equation}
	v_{extrap} = \frac{2^p v_{new} - v_{old} }{2^p - 1}.
\end{equation}

This can be repeated a second time in order to get a new extrapolated value\footnote{We have not obtained a better approximation but the confirmation of at lteast $4$ digits (42.0495) when $I=J=1536$ and of six digits (42.049561) when $I=J=24576$.}, but this time with $p=1$, because the order is $1$ after one extrapolation, as we can see in Table \ref{table2}. Finally, we have kept the values of the contract as $I, J$ increase and after two extrapolations (with $p=1$) in Table \ref{table3}, with $S_{max} = 8X$\footnote{Other values of $S_{max}$, like $4X$, $16X$ and $32X$, have been tried, but none of them has provided a better convergence ratio than $8X$, so we have decided to include only a table for $S_{max}=8X$ in order to not being repetitive.}.

\begin{table}[h!]\scriptsize
	\setlength{\tabcolsep}{15pt}
	\renewcommand{\arraystretch}{1.2}
	\begin{tabular}{rrrrl}
	I& J&   $V(S_0,0)$		& 1st Extrapolation & 2nd Extrapolation\\\hline
	   48 &    48 & 42.0048918538 & 42.0595720633 & -                                     \\
	96 &    96 & 42.0412438594 & 42.0533611946 & 42.047150325933316 \\
	192 &   192 & 42.0486740302 & 42.0511507538 & 42.048940312999996  \\
	384 &   384 & 42.0497380652 & 42.0500927435 & 42.04903473326667  \\
	768 &   768 & 42.0496093308 & 42.0495664193 & 42.04904009513334   \\
	1536 &  1536 & 42.0495722630 & 42.0495599071 & 42.04955339480001    \\
	3072 &  3072 & 42.0495647830 & 42.0495622897 & 42.04956467226666   \\
	6144 &  6144 & 42.0495620065 & 42.0495610810 & 42.04955987233335    \\
	12288 & 12288 & 42.0495617211 & 42.0495616260 & 42.04956217093333  \\
	24576 & 24576 & 42.0495614284 & 42.0495613308 & 42.049561035700016   \\
	49152 & 49152 & 42.0495614516 & 42.0495614593 & 42.04956158783331  \\
	\end{tabular}
	\vspace{0.2cm}
	\captionsetup{width=.55\linewidth}
	\caption{Values when increasing $\texttt{m_I}$ and $\texttt{m_J}$, Lagrange Interpolation of degree $16$ and $S_{max} = 8X$.}\label{table3}
\end{table}
\vspace{-0.5cm}
From this table we can conclude that the $V(S_0,0)$ converges to $42.049561$ as $I$ and $J$ increases. However, when $I = J \geq 16384$, the time consumption is too high. To reduce it, we just can get three values using $I= J \in \{3072,6144\}$ and do the extrapolation of order $2$. Since $3072+6144 = 9216 < 12288$, we can obtained six digits of precision with 3072 iterations less and in $8.979847 + 36.19612 = 45.175967$s, $126.9858$s less than when using $I = J = 12288$, whose result is obtained in $172.1617889$s (extrapolation time is near 0).

If we would want to have just five digits of precision, we could either use $I=J= 3072$ with no extrapolation (8.979847s) or $I=J=384$ (0.1666s) and $I=J=568$ (0.6226s), i.e. 952 iterations (2120 less iterations)  and 0.7892s (less than 1s).

To sum up, we can see in Table \ref{table4} the best results we've got when using $S_{max}\in\{8X,12X\}$. We can definitely say that $V(S,t=0) \approx 42.095$, and that $V(S,t=0) \approx 42.0956\pm 10^{-5}$. To obtain this result, the fastest way is to use $N=1280$, $S_{max} = 12X$ and two extrapolations.
\begin{table}[h!]
	\setlength{\tabcolsep}{12pt}
	\renewcommand{\arraystretch}{1.25}
	\begin{tabular}{c||cc||cc}
		 I, J		&  	Value & Time (s)& Extrapolated Value& Time (s)\\\hline\addlinespace[0.05cm]
		952 	& 	42.04953 & 1.3855&42.04956 	& 0.7892 \\
		9216 	&	42.049560& 106.5886&42.049561 	& 126.9858 \\
	\end{tabular}
\vspace{0.25cm}
	\captionsetup{width=.639\linewidth}
	\caption{Values of the contract when using directly Crank Nicolson and using two different values to obtain the extrapolated value with $p=2$.}\label{table4}
\end{table}
\vspace{-0.5cm}
Finally, just mentioned that ratios of the differences between values may give us the convergence rate, which is theoretically $O((\Delta t)^2, (\Delta S)^2)$. Since they are the same and we have increases of $2n$, we expect to have ratios around $4$. For example, for $S_{max}$, the ratios we've obtained are given in Table \ref{table2}. As we can see, the ratios are not always $4$, so we may not expect the extrapolatino to be perfect. However, as we have mentioned before, many ratios are near to $4$, and indeed the extrapolation has provide us with good result, as we can see in Table \ref{table4}
