In this section we are trying to find the most possible accurate value for $V(17.38,0)$ and further obtain in less than one second. To do so, we are going to repeat the same procedure than in Section 1, that is, increase $I$ and $J$ in order to study the convergence rate of Crack Nicolson method in our particular situation, which is \textit{a priori} $\mathcal{O}\left((\Delta S)^2, (\Delta t)^2\right)$. The main difference here is that not only we must try to get $S_0$ within the (vertical) grid points to reduce eventual interpolation errors but also $t_0 = 0.96151$ (within the horizontal axis). 
\begin{table}[h!]\scriptsize
	\setlength{\tabcolsep}{15pt}
	\renewcommand{\arraystretch}{1.2}
	\begin{tabular}{cclll}
		$\texttt{m_I}$ & $\texttt{m_J}$& Without Extrap.&Ratios (\texttt{diffOld/diff}) & One Extrap. (deg.=2)\\ \hline\addlinespace[0.2cm]
		32 	&    16 & 42.7226015285 & &43.398548157933334                  \\
		64 	&    32 & 43.079752062 &5.677829649105139& 43.19880223983333   \\
		128 &   64 & 43.18821969729999 & 3.292692170454403& 43.224375575733326  \\
		256 &   128 & 43.2219325264 & 3.217399375715954& 43.23317013610001  \\
		512 &   256 &  43.230133027600004 &4.111069345372084 &43.23286652800001      \\
		1024 &  512 & 43.2318990171 &4.6435730223934675 & 43.232487680266665 \\
		2048 &  1024 & 43.2327402217 & 2.0993578732129685&43.23302062323333  \\
		4096 &  2048 & 43.233638732399996 & 0.9362210155126094&43.23393823596666   \\
		8192 & 4096 & 43.2338258054 & 4.8029950872996325&43.23388816306667   \\
		16384 & 8192 & 43.2337964887 & 6.381107014103734&43.23378671646666   \\	
	\end{tabular}
	\vspace{0.4cm}
	\captionsetup{width=.5\linewidth}
	\caption{Values when increasing $\texttt{m_I}$ and $\texttt{m_J}$, Lagrange Interpolation of degree $16$ and $S_{max} = 8X$.}\label{table6}
\end{table}
To do so, we must note that there must exist two integers $j^*$ and $i^*$ such that $j^* \texttt{m_dS = S0}$ and $i^* \texttt{m_dt = t_0}$. This implies that $\texttt{m_J} = j^* \texttt{Smax / S0}$ and $\texttt{m_I} = i^* \texttt{T} / t_0$. In our case, $X = S_0$, so we can set $\texttt{Smax = k*X}$ for some integer $k$, and therefore $\texttt{m_J} = \texttt{n * ceil(Smax/S0) = n*k}$, for any integer $\texttt{n}$. We have not such a relation between $\texttt{T}$ and $t_0$, because it's fixed. Therefore, set $\texttt{m_I} =  \texttt{n * ceil(\texttt{T}/t_0)}$, and, although it might not be exactly on the grid, it would not be very far from being on it. Hence, we can try a loop $\texttt{for (int n=nMin; n<=nMax; n*=2)}$ in order to get the Table \ref{table6}.

As we can see in this table, the ratio of the convergence\footnote{We increase $n$ by doubling it each lap. Since the convergence is $\mathcal{O}\left((\Delta S)^2, (\Delta t)^2\right)$, if we double $n$, the value would increase as a ratio of $4$ with respect to the previous difference, i.e., if we take three values $v_m, v_{2m}, v_{4m}$ with $m\equiv 0  \text{ mod } n$,then
$$
	\frac{v_{2n} - v_{n}}{v_{4n} - v_{2n}}\approx 2^2 = 4.
$$} is around $2$, although the monotony is pointly broken. It may be caused by a higher disparity between the new result and the previous one (the greater the denominator, the smaller the result). On the contrary, we get 5 or 6 when the values are similar (opposite~situation). Nevertheless, the ratios are around $4$, so we can explore using Richardson extrapolation ($p=2$) to see whether we can improve our results with less iterations.

Generally speaking, we can see that $V(S_0,0)$ converges to $43.2337\pm10^{-4}$, i.e., we can only ensure three digits of precision (in a reasonable time), i.e., $V(S_0,0)\approx 42.233$. On the other hand, as we can see in Table \ref{table6}, we just need to run our code for $\texttt{(m_I, m_J)} = (128, 64)$ and $\texttt{(m_I, m_J)} = (256, 128)$. 

\vspace{-0.1cm}
\begin{table}[h!]\small
	\setlength{\tabcolsep}{18pt}
	\renewcommand{\arraystretch}{1.15}
	\begin{tabular}{ccll}
		$\texttt{m_I}$ & $\texttt{m_J}$& $V(S_0,0)$ & Time (s)\\ \hline\addlinespace[0.2cm]
		128 &   64 & 43.18821969729999 &  0.0107786  \\
		256 &   128 & 43.2219325264 &  0.039997  \\\addlinespace[0.2cm]\hline\hline\addlinespace[0.1cm]
		\multicolumn{2}{c}{Extrap. Value} & 43.23317013610001 & $9.999 \times 10^{-6}$ \\\addlinespace[0.1cm]\hline\hline
	\end{tabular}
	\vspace{0.4cm}
	\captionsetup{width=.5\linewidth}
	\caption{Obtaining $V(S_0,0)$ with three digits of precision in less than one second.}\label{table7}
\end{table}
\vspace{-0.6cm}
The extrapolated value has been obtain by a simple code which can be found in \texttt{GeneralFunctions.cpp}. Doing that, we obtaing that $V(S_0,0)\approx 42.233$ in 0.05758599 s.

